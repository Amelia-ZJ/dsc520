\documentclass[]{article}
\usepackage{lmodern}
\usepackage{amssymb,amsmath}
\usepackage{ifxetex,ifluatex}
\usepackage{fixltx2e} % provides \textsubscript
\ifnum 0\ifxetex 1\fi\ifluatex 1\fi=0 % if pdftex
  \usepackage[T1]{fontenc}
  \usepackage[utf8]{inputenc}
\else % if luatex or xelatex
  \ifxetex
    \usepackage{mathspec}
  \else
    \usepackage{fontspec}
  \fi
  \defaultfontfeatures{Ligatures=TeX,Scale=MatchLowercase}
\fi
% use upquote if available, for straight quotes in verbatim environments
\IfFileExists{upquote.sty}{\usepackage{upquote}}{}
% use microtype if available
\IfFileExists{microtype.sty}{%
\usepackage[]{microtype}
\UseMicrotypeSet[protrusion]{basicmath} % disable protrusion for tt fonts
}{}
\PassOptionsToPackage{hyphens}{url} % url is loaded by hyperref
\usepackage[unicode=true]{hyperref}
\hypersetup{
            pdftitle={8.3 Final Project Step 1},
            pdfauthor={Amelia Farrell},
            pdfborder={0 0 0},
            breaklinks=true}
\urlstyle{same}  % don't use monospace font for urls
\usepackage[margin=1in]{geometry}
\usepackage{graphicx,grffile}
\makeatletter
\def\maxwidth{\ifdim\Gin@nat@width>\linewidth\linewidth\else\Gin@nat@width\fi}
\def\maxheight{\ifdim\Gin@nat@height>\textheight\textheight\else\Gin@nat@height\fi}
\makeatother
% Scale images if necessary, so that they will not overflow the page
% margins by default, and it is still possible to overwrite the defaults
% using explicit options in \includegraphics[width, height, ...]{}
\setkeys{Gin}{width=\maxwidth,height=\maxheight,keepaspectratio}
\IfFileExists{parskip.sty}{%
\usepackage{parskip}
}{% else
\setlength{\parindent}{0pt}
\setlength{\parskip}{6pt plus 2pt minus 1pt}
}
\setlength{\emergencystretch}{3em}  % prevent overfull lines
\providecommand{\tightlist}{%
  \setlength{\itemsep}{0pt}\setlength{\parskip}{0pt}}
\setcounter{secnumdepth}{0}
% Redefines (sub)paragraphs to behave more like sections
\ifx\paragraph\undefined\else
\let\oldparagraph\paragraph
\renewcommand{\paragraph}[1]{\oldparagraph{#1}\mbox{}}
\fi
\ifx\subparagraph\undefined\else
\let\oldsubparagraph\subparagraph
\renewcommand{\subparagraph}[1]{\oldsubparagraph{#1}\mbox{}}
\fi

% set default figure placement to htbp
\makeatletter
\def\fps@figure{htbp}
\makeatother


\title{8.3 Final Project Step 1}
\author{Amelia Farrell}
\date{October 25th 2021}

\begin{document}
\maketitle

\subsection{Topic: Supermarket Shrink - How supermarkets can reduce
waste by ordering the right amount at the right
time.}\label{topic-supermarket-shrink---how-supermarkets-can-reduce-waste-by-ordering-the-right-amount-at-the-right-time.}

The problem with shrinkage in suparmarkets can be seen in every
department. From trasport to checkout, dariy to cearel. According to
research by the FMI and The Retail Control Group, 64\% of store shrink
can be traced back to ineffective store operating practices
(wheresmyshrink.com, 2012). The highest percentage of this comes from
Ordering and Production Planning inefficiencies. Meaning that if we can
simply order better and plan to stock the right items, we could reduce
64\% of supermarket shrink! However, we know it is not as simple as
that. We need to break this large task into smaller more managle pecies
in order to build ourselves a starting place to solving this wide spread
problem. Meat and Produce the largest contitubors to shrink from
parishable goods. For the purposes of this assignment (and my own
troubles with keeping purchacing friut right before it goes bad), we
will be foucusing on the shrink steming from the produce section of the
standard American supermarket.

\subsection{Research questions}\label{research-questions}

\begin{enumerate}
\def\labelenumi{\arabic{enumi}.}
\tightlist
\item
  How much does the avarge grocry store lose in fresh friut/vegables per
  year (in \$ value and pounds)?
\item
  What are the cuases (5 Whys) behind the high shrikage.waste of produce
  in grocrsy stores?
\item
  What produce produces the most amount of waste (only waste occring
  within the supermarket, does not include farming, transit, customer
  waste)?
\item
  How are the high waste items corrilated with one another?
\item
  Is there any relationship to the \% of waste (pounds of produce lost)
  and the time of year? Sesonality.
\item
  Are there any corrilations between the amount of produce sold and
  other non-pershible goods? E.g. does ceriel sales increase at the same
  rate as bananas?
\item
  Can we make any predictions on what and how much produce to stock
  based off other predictors?
\item
  How can we use the information we gathered from the above questions to
  make better stocking decisens and reduce shrikage for produce?
\end{enumerate}

\subsection{Probem Statment: How can we make better predictions on the
right aoumnt of produce to purchase to maxamize sales and decearse
waste?}\label{probem-statment-how-can-we-make-better-predictions-on-the-right-aoumnt-of-produce-to-purchase-to-maxamize-sales-and-decearse-waste}

\subsection{References}\label{references}

Field, A., J. Miles, and Z. Field. 2012. Discovering Statistics Using R.
SAGE Publications. \url{https://books.google.com/books?id=wd2K2zC3swIC}.

\end{document}
